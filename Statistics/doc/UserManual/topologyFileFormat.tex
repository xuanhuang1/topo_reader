\section{Topology File Format} \label{sec:topologyFileFormat} 
Up to three files per timestep comprise the topology file format:

\begin{enumerate}
\item {\bf Family Files:} contain a list of features, their lifetime information and feature properties.
\item {\bf Segmentation Files:} contain a list of domain elements for each feature.
\item {\bf Map Files:} $\left<optional\right>$ contains a list of indices into a global index map.  
\end{enumerate}

\paragraph{Family Files}
Family files encode the feature hierarchy and store attributes associated with the features.  
Because the number of attributes stored is arbitrary, there is an XML-like footer that contains 
the information necessary to parse the file, see for example figure \ref{fig:footer}.  The footer
contains handles pointing to the simplification hierarchy in the file, as well as details on
and pointers to each of the statistcal attributes.  The following statistics 
are currently supported: vertex count, function min, function max, integrated function value, and descriptive statistics 
(first through fourth order moments).  We are in the process of adding support for contingency statistics, which includes
point-wise mutual information and information entropy.  

Statistical attributes can be stored pre-accumulated or on a per feature basis.  
When statistics are stored on a per feature basis, the topological analysis codes have 
aggregated the statistics for each feature individually and written the results to file.  
The topological analysis codes can also pre-accumulate statistics accordig to the simplification 
sequence, greatly accelerating post-processing analyses in downstream tools.  When pre-accumulated 
statistics have been written to the file, the aggregated option is set to yes in 
the Statistic XML handle.  

\begin{figure}[ht]
\tiny{
\begin{verbatim}
<Clan offset="2033857" globalsize="4" localsize="4" precision="float" major="1" minor="0">
  <Family offset="0" range="1.0000021756e-01 1.9924249649e+01">
      <Simplification offset="0" encoding="ascii" elementcount="16422" metric="High Threshold" filetype="0" range="1.0000021756e-01 1.9924249649e+01"/>
      <Statistic offset="445056" encoding="ascii" elementcount="16422" mapping="direct" aggregated="yes" attribute="temp" statistic="vertexCount"/>
      <Statistic offset="486845" encoding="ascii" elementcount="16422" mapping="direct" aggregated="yes" attribute="H2_ConsumptionRate" statistic="mean"/>
      <Statistic offset="678906" encoding="ascii" elementcount="16422" mapping="direct" aggregated="yes" attribute="H2_ConsumptionRate" statistic="min"/>
      <Statistic offset="835438" encoding="ascii" elementcount="16422" mapping="direct" aggregated="yes" attribute="H2_ConsumptionRate" statistic="max"/>
      <Statistic offset="993375" encoding="ascii" elementcount="16422" mapping="direct" aggregated="yes" attribute="temp" statistic="mean"/>
      <Statistic offset="1185414" encoding="ascii" elementcount="16422" mapping="direct" aggregated="yes" attribute="temp" statistic="min"/>
      <Statistic offset="1341916" encoding="ascii" elementcount="16422" mapping="direct" aggregated="yes" attribute="X(OH)" statistic="mean"/>
      <Statistic offset="1578987" encoding="ascii" elementcount="16422" mapping="direct" aggregated="yes" attribute="x" statistic="mean"/>
      <Statistic offset="1729844" encoding="ascii" elementcount="16422" mapping="direct" aggregated="yes" attribute="y" statistic="mean"/>
      <Statistic offset="1880301" encoding="ascii" elementcount="16422" mapping="direct" aggregated="yes" attribute="z" statistic="mean"/>
  </Family>
</Clan>
\end{verbatim}
}
\label{fig:footer}
\caption{Sample XML footer for a family file.}
\end{figure}


\paragraph{Segmentation Files}
Segmentation files are typically binary files that contain a list of domain elements associated with each feature.  
The file is written as a flat array of one index per element, with the index corresponding to its feature id.

\paragraph{Map Files}
Map files are optional files that are typically binary. The file is written as a flat array of one index per 
element indicating the index of this element with respect to a virtual index space.

