\documentclass[journal]{vgtc}         % review (journal style)

\usepackage{mathptmx}
\usepackage{graphicx}
\usepackage{times}
\usepackage{amsthm}
\usepackage{amsfonts}

\def\Rspace{\mathbb{R}} % ptb
\def\Aspace{\mathbb{A}} % ptb
\def\Nspace{\mathbb{N}} % ptb
\def\Mspace{\mathbb{M}} % ptb
\def\Fspace{\mathbb{F}} % ptb
\def\Cspace{\mathcal{F}} % ptb
\def\Pset{\mathcal{P}} % ptb



\newtheorem{definition}{Definition}

\title{Feature Family Processing}

\abstract{ This document describes some general thoughts and design decision
  relavant to the fast processing of feature families. For now we asssume we are
  interested in processing a {\it clan} of one-parameter feature families
  computed from a number of time steps of a simulation. We will describe the
  general concepts, define the necessary operators and discuss their
  implementation wrt. the latest data structures.  }


\begin{document}

\firstsection{Generic Concepts}
\maketitle

First, we describe the processing pipeline and it elements in a generic way
independent of implementation issues.

\subsection{Definitions}

The source data we are working on is a {\it clan}, $C=\{F_0,....., F_n\}$, of
{\it feature families} $F_i$ for time steps $t=\{0,...,n\}$. Each feature family
is dependent on {\it parameters} $p^1,...,p^m$ to define a set of {\it
  features}, $\{f_0^i,...,f_{n_i}^i\}$. We call the space of feature families
$\Cspace$ and the space of possible features $\Fspace$.

\begin{definition}[Clan]
  A \emph{clan} $C$ is an ordered set of feature families
  
  $$C = \{F_0,....., F_n\} \subset \Cspace$$
\end{definition}

\begin{definition}[Feature Family]
  A \emph{feature family} $F$ is a set of features 

  $$F = \{f_0,....,f_m\} \subset \Fspace$$

\end{definition}

\begin{definition}[Feature]
  A \emph{feature} $f$ is a collection of parameter ranges $[p_{min},p_{max}]$
  together with a set of statistics $\{stat^0,...stat^k\}$:

  $$f =
 \left(\left\{[p_{min}^1,p_{max}^1],...,[p_{min}^m,p_{max}^m]\right\},\left\{stat_0,...stat_k\right\}\right)
 \in \Fspace$$
\end{definition}


\subsection{Operators}

There are four types of operators mapping families, sets of features, and
statistics into new sets of features, sets of scalars, or numbers.

\begin{definition}[Selection]
  
  A \emph{selection} $S: \Cspace \times \Rspace^m \rightarrow \Pset(\Fspace)$
  is a function that maps a feature family and a list of parameters to a set of
  features. 

\end{definition}

\begin{definition}[Subselection]

  A \emph{subselection} $U: \Pset(\Fspace) \times \{0,...,k\} \times \Rspace^2
  \rightarrow \Pset(\Fspace)$ is a function that for a set of features and an
  index to a statistics returns a set containing only those features for which
  the statistic value lies within the given interval.

\end{definition}

\begin{definition}[Accumulation]

  An \emph{accumulation} $A: \Pset(\Fspace) \times \{0,...,k\} \rightarrow
  \Pset(\Rspace)$ is a function that for a set of features and a statistics
  index return the set of corresponding statistics.

\end{definition}

\begin{definition}[Reduction]

  A \emph{reduction} $R: \Pset(\Rspace) \rightarrow \Rspace$ is a function that
  maps a set of scalar values to a single scalar value, for examaple by
  computing the mean.

\end{definition}

\section{Graph Types}

We want to use the definitions and operators decsribed above to create three
different types of plots: Species distributions; Parameter graphs; and Time
plots. Here we will briefly describe each type and indicate how they could be
implemented with the operators given above. For simplicity of presentation we
assume that all functions operate on the entire clan even though in practice one
could easily restrict it to a temporal subset

\subsection{Species Distributions}

These are graphs such as histographs, cummulative density functions (CDFs), or
weighted CDFs collecting information of all features for a given choice of
parameters and subselections

\paragraph{Inputs:} 
\begin{enumerate}
\item A fetaure clan $C$;
\item A list of parameters $P = (p^0,...,p^m)$; 
\item A set of subselections $Q$ = $\{(stat_{min}^{i_0},stat_{max}^{i_0}), ... ,
  (stat_{min}^{i_l},stat_{max}^{i_l})\}$ with $l \le k$; and
\item A statistic index $i$
\end{enumerate}

\paragraph{Algorithm:}
\begin{tabbing}
\=\hspace{0.2in}\=\hspace{0.2in}\=\hspace{0.2in}\=\hspace{0.2in}\=\\
\>{\sc SpeciesDistribution}($C$,$P$,$Q$,$i$ ) \\
\>\>$Output = \emptyset$\\
\>\>for $F \in C$\\
\>\>\>$X$ = $S(F,P)$\\
\>\>\>for $j \in \{i_0,...,i_l\}$\\
\>\>\>\>$X$ = $U(X,j,(stat_{min}^{j},stat_{max}^{j}))$\\
\>\>\>$Output$ = $Output \cup A(X,i)$\\ 
\>\>{\sc Sort}($Output$)
\end{tabbing}


\paragraph{Discussion:}

The first selection step should be fast and is unavoidable. Similarly, the last
accumulation to return the final values to display is unavoidable. The expensive
portion of this algorithm is the middle portion where additional accumulations
must be performed for each subselection. In practice, it might be faster to
first compute all statistics used in the subselection at once (potentially even
together with the final one) and reduce the set based on these values


\subsection{Parameter Graphs}

These are graphs that show how a quantity of interest changes as one parameter
is varied. The most common example is a plot of average number of features per
family per parameter.

\paragraph{Input}
\begin{enumerate}
\item A fetaure clan $C$;
\item A list of parameters $P = (p^0,...,p^{j-1},p^{j+1},...,p^m)$; 
\item A list of values for the parameter of interest $V = (p^j_0,...,p^j_h)$;
\item A set of subselections $Q$ = $\{(stat_{min}^{i_0},stat_{max}^{i_0}), ... ,
  (stat_{min}^{i_l},stat_{max}^{i_l})\}$ with $l \le k$; 
\item A statistics index $i$;
\item A family wide reduction $R_f$; and
\item A clan wide reduction $R_c$.
\end{enumerate}

\paragraph{Algorithm:}
\begin{tabbing}
\=\hspace{0.2in}\=\hspace{0.2in}\=\hspace{0.2in}\=\hspace{0.2in}\=\\
\>{\sc ParameterGraph}($C$,$P$,$V$,$Q$,$i$,$R_f$,$R_c$ ) \\
\>\>$Output = \{\emptyset,...,\emptyset\}$\\
\>\>for $F \in C$\\
\>\>\>for $p \in V$\\
\>\>\>\>$X$ = $S(F,P\cup p)$\\
\>\>\>\>for $j \in \{i_0,...,i_l\}$\\
\>\>\>\>\>$X$ = $U(X,j,(stat_{min}^{j},stat_{max}^{j}))$\\
\>\>\>\>$Output[p] = Output[p] \cup R_f(A(X,i))$\\
\>\>for $o \in Output$\\
\>\>\>$o = R_c(o)$
\end{tabbing}

\paragraph{Discussion}

This is an even more expensive algorithm since you have to adapt the statistics
for each $p$-value independently.


\subsection{Time Plots}

A time plot as the name suggests shows the evolution of a scalar quantity. Since
time corresponds to the order of feature families this is fairly straight
forward.

\paragraph{Input:}

\begin{enumerate}
\item A fetaure clan $C$;
\item A list of parameters $P = (p^0,...,p^m)$; 
\item A set of subselections $Q$ = $\{(stat_{min}^{i_0},stat_{max}^{i_0}), ... ,
  (stat_{min}^{i_l},stat_{max}^{i_l})\}$ with $l \le k$; 
\item A statistics index $i$; and
\item A family wide reduction $R_f$; 
\end{enumerate}


\paragraph{Algorithm:}
\begin{tabbing}
\=\hspace{0.2in}\=\hspace{0.2in}\=\hspace{0.2in}\=\hspace{0.2in}\=\\
\>{\sc ParameterGraph}($C$,$P$,$V$,$Q$,$i$,$R_f$,$R_c$ ) \\
\>\>$Output = [0,...,0]$\\
\>\>for $F \in C$\\
\>\>\>$X$ = $S(F,P)$\\
\>\>\>for $j \in \{i_0,...,i_l\}$\\
\>\>\>\>$X$ = $U(X,j,(stat_{min}^{j},stat_{max}^{j}))$\\
\>\>\>$Output[F] = R_f(A(X,i))$\\
\end{tabbing}

\paragraph{Discussion}

This is a simpler version of the parameter graph. 


\section{Practical Implementation}

Some thoughts on the curent implementation:

\begin{itemize}
\item The selection function is a state rather than a true function. This makes
  updating the selection for successive parameters cheap
\item It would be useful to iterate through the constituents of any feature
\item If the statistics must be computed we might be better off computing all of
  them in one traversal rather than successively downselect.
\end{itemize}

\end{document}
