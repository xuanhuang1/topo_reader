\documentclass[]{article}
\usepackage{graphicx}
\usepackage{amsmath}
\usepackage{graphics}
\usepackage{latexsym}
\usepackage{amssymb}
\usepackage{graphics}
\usepackage{color}
\usepackage{subfigure}
\usepackage{wrapfig}

\title{Algorithm Notes for Merge Tree Computation}
\author{Peer-Timo Bremer}
\date{}

\begin{document}
\maketitle

\section{Merging of Multiple Merge Trees}

{\bf Assumption:}
\begin{itemize}
\item The merger can determine whether an incoming node is on the boundary
\item The merger can determine how many instances of a node will arrive
  (assuming correct computation upstream)
\end{itemize}

\noindent
{\bf Input:} 
\begin{itemize}
\item $k$ streams of nodes, arcs, and finalization information describing $k$ merge trees
\item The streams are asynchronous and not guaranteed to be in any order even
  though practically speaking they will be somewhat sorted
\end{itemize}

\noindent
{\bf Algorithm:}
\begin{enumerate}
\item Parse the input string to
  \begin{enumerate}
  \item[(a)] Delay finalization information until all instances of a vertex have
    been seen
  \item[(b)] Mark boundary nodes
    \item[(c)] Store the incoming arcs as node pairs
  \end{enumerate}
\item Compute combined tree using the enhanced streaming algorithm (sorted branch
  decomposition). The algorithm produces a stream of nodes, arcs, and
  finalization information building the resulting tree.
\item Store the resulting tree locally 
\item Each time an arc is finalized determine wether it remains part of the
  boundary is so output it to the next layer, if not do nothing
\item For all original arcs take the stored combined tree and 
  \begin{enumerate}
  \item[(a)] If the top node is no longer critical determine with which arc it merged and
    store it
  \item[(b)] If the top node is still critcal then either its old bottom node is
    still its representative or a split has taken place. Store all the splits
    above the old bottom node and their function values. 
  \end{enumerate}
\item Send of the delta changes to the respective incoming compute nodes
\end{enumerate}

\noindent
{\bf Thoughts and Comments:}
\begin{itemize}
\item If each local merge node stores its final tree than it can figure out a
  delta encoding between its final tree and the final tree after the next
  level. 
\item The above technique will lead to a cascading communication pattern in
  which each round of merging kicks of more communication
\item Furthermore, we will send potentially much smaller messages but we'll send
  more of them 
\item How do we best store a tree to allow fast delta encoding (which is the
  same operation as fixing the segmentation). Do we care ? We could do all
  merges (simple lookup whether a node exists) and then all splits in more or
  less optimal time ?
\end{itemize}

\noindent
{\bf Alternatives:}
\begin{itemize}
\item Send of the resulting tree instead of the delta information
\item Send either the delta or the resulting tree directly to the corresponding
  leafs
\end{itemize}

\end{document}