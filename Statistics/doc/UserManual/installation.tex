\section{Installation} \label{sec:installation}

At the time of writing this document, there are three software 
components that have been developed to analyze and visualize feature families.  
These include a Statistics browser, a Merge Tree viewer, and a Segmentation
viewer.  All three of these codes rely on the TopologyFileParser library.  The Segmentation
viewer relies on the FlexArray library as well as the RenderUtility library(for high quality renderer).


The code can be checked out from SCI's svn repository: 

\small{
\begin{description}
\item[Topology File Parser Library:]https://gforge.sci.utah.edu/svn/dar/topology/TopologyFileParser

\item[Flex Array Library:]https://gforge.sci.utah.edu/svn/dar/topology/FlexArray

\item[Render Utility Library:]https://gforge.sci.utah.edu/svn/dar/renderUtility

\item[Segmentation Viewer:]https://gforge.sci.utah.edu/svn/dar/topology/SegViewer (low quality rendering)

\item[Segmentation Viewer:]https://gforge.sci.utah.edu/svn/dar/topology/SegViewer/VolRender (high quality rendering)

\item[Merge Tree Viewer:]https://gforge.sci.utah.edu/svn/dar/topology/TrackingGraph

\item[Statistics Browser:]https://gforge.sci.utah.edu/svn/dar/topology/Statistics
\end{description}
}

All of the code is meant to be installed in an ``out of source'' manner using CMake, see the
CMake home page at \url{http://cmake.org} for details on the CMake system.  

\paragraph{Flex Array Library} 
This library is required by the Segmentation Viewer and provides flexible array functionality, 
including an out of core implemenation.  To install, create your build directory and from there
type:

\texttt{\% ccmake /path/to/source}

to generate the necessary makefiles. Once you have exited \texttt{ccmake}, type 

\texttt{\% make}\\
\texttt{\% make install}

to build and install this library.


\paragraph{Topology File Parser Library} 
This library implements the topology file format and is required by all three software 
components.  To install, create your build directory and from there type:

\texttt{\% ccmake /path/to/source}

to generate the necessary makefiles. Once you have exited \texttt{ccmake}, type 

\texttt{\% make}\\
\texttt{\% make install}

to build and install this library.

\paragraph{Render Utility Library} 
This library implements the rendering helper functions and is required by the high quality version of the SegViewer tool.
To install, create your build directory and from there type:

\texttt{\% ccmake /path/to/source}

to generate the necessary makefiles. Once you have exited \texttt{ccmake}, type 

\texttt{\% make}\\
\texttt{\% make install}

to build and install this library.

\paragraph{Segmentation Viewer (Low quality renderer - for older machines)} 
The Segmentation viewer is a software component that visualizes the feature segmentation.
To install, create your build directory and from there type:

\texttt{\% ccmake /path/to/source}

to generate the necessary makefiles. 

You will need to modify the 
\texttt{FLEXARRAY\_INCLUDE\_DIR} and \texttt{FLEXARRAY\_LIBRARIES} variables to point
to your installation of the FlexArray include directory and library respectively.
You will also need to modify the \texttt{TOPOLOGY\_FILE\_PARSER\_INCLUDE\_DIR} and 
\texttt{TOPOLOGY\_FILE\_PARSER\_LIBRARIES} variables to point to your installation of the 
TopologyFileParser include directory and library respectively.

This software component has a gui that relies on fltk2.  If you do not have this installed
on your machine you can download it from \url{http://fltk.org}.  You may need to modify the associated 
\texttt{FLTK2} variables to point your installation of fltk2.

This software component also relies on GLEW.  If you do not have this installed
on your machine you can download it from \url{http://glew.sourceforge.net/}.  You may need to 
modify the associated \texttt{GLEW} variables to point your installation of GLEW.

Once you have exited \texttt{ccmake}, type 

\texttt{\% make}\\
\texttt{\% make install}

to build and install this library.
 
\paragraph{Segmentation Viewer (High quality renderer - for newer machines)} 
This version requires a high end GPU. Preferrably of Fermi class (nVidia GTX460 or better).

\texttt{\% ccmake /path/to/source}

to generate the necessary makefiles. 

You will need to modify the 
\texttt{FLEXARRAY\_INCLUDE\_DIR} and \texttt{FLEXARRAY\_LIBRARIES} variables to point
to your installation of the FlexArray include directory and library respectively.
You will also need to modify the \texttt{TOPOLOGY\_FILE\_PARSER\_INCLUDE\_DIR} and 
\texttt{TOPOLOGY\_FILE\_PARSER\_LIBRARIES} variables to point to your installation of the 
TopologyFileParser include directory and library respectively.

This software component has a gui that relies on fltk2.  If you do not have this installed
on your machine you can download it from \url{http://fltk.org}.  You may need to modify the associated 
\texttt{FLTK2} variables to point your installation of fltk2.

This software component also relies on GLEW.  If you do not have this installed
on your machine you can download it from \url{http://glew.sourceforge.net/}.  You may need to 
modify the associated \texttt{GLEW} variables to point your installation of GLEW.

This tool also depends on the RenderUtility library. You will need to modify
\texttt{RENDERUTILITY\_INCLUDE\_DIR} and 
\texttt{RENDERUTILITY\_LIBRARIES} variables to point to your installation of the 
RenderUtility include directory and library respectively.

Once you have exited \texttt{ccmake}, type 

\texttt{\% make}\\
\texttt{\% make install}

to build and install this library.


\paragraph{Merge Tree Viewer} 
The Merge Tree viewer displays a merge tree as a branch decomposition.  
To install, create your build directory and from there type:

\texttt{\% ccmake /path/to/source}

to generate the necessary makefiles. 

You will need to modify the \texttt{TOPOLOGY\_FILE\_PARSER\_INCLUDE\_DIR} and \texttt{TOPOLOGY\_FILE\_PARSER\_LIBRARIES} 
variables to point to your installation of the TopologyFileParser include directory and 
library respectively.

This software component has a gui that relies on fltk2.  If you do not have this installed
on your machine you can download it from \url{http://fltk.org}.  You may need to modify the associated 
\texttt{FLTK2} variables to point your installation of fltk2.

Once you have exited \texttt{ccmake}, type 

\texttt{\% make}\\
\texttt{\% make install}

to build and install this library.


\paragraph{Statistics Browser} 
The Statistics Browser displays plots including species distribution plots, time plots and 
parameter plots (described in section \ref{sec:featureFamilyProcessing}. Sliders are provided
so that the user can subselect the features according to various criterion.

To install, create your build directory and from there type:

\texttt{\% ccmake /path/to/source}

to generate the necessary makefiles. 

You will need to modify the 
\texttt{TOPOLOGY\_FILE\_PARSER\_INCLUDE\_DIR} and \texttt{TOPOLOGY\_FILE\_PARSER\_LIBRARIES} 
variables to point to your installation of the TopologyFileParser include directory and 
library respectively.

This software component has a gui that relies on the stable release of fltk (at the time of writing this
is 1.1.10).  If you do not have this installed on your machine you can download it from 
\url{http://fltk.org}.  You may need to modify the associated \texttt{FLTK} variables to point your installation of fltk.

This software component also uses the FreeType2 and FTGL libraries for display of fonts and text.  If you do not have
these installed on your machine, you can download them from \url{http://freetype.sourceforge.net/download.html} 
and \url{http://sourceforge.net/projects/ftgl/}.  You may need to modify the associated \texttt{FREETYPE2} 
and \texttt{FTGL} variables to point to your installation of FreeType2 and FTGL respectively.

Once you have exited \texttt{ccmake}, type 

\texttt{\% make}\\
\texttt{\% make install}

to build and install this library.

