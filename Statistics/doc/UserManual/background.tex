\section{Background}
\label{sec:background}

Morse theory is one way to study the topology of scalar fields.
For a survey of other applications of topology to data analysis, see
Carlsson~\cite{car09}.

\paragraph{Notation}
Morse theory analyzes the topology of a manifold  $\mathbb{M}$ by studying
differentiable generic functions defined on $\mathbb{M}$
\cite{Milnor:Morse,Matsumoto:Morse}.  Let $\mathbb{M}$ be a smooth, compact
$d$-manifold without boundary and let $f:\mathbb{M}\rightarrow \mathbb{R}$
denote a smooth real-valued function on $\mathbb{M}$.  Assuming a local
coordinate system at a point $x\in\mathbb{M}$, the \emph{gradient} $\nabla
f(x)$ consists of all partial derivatives and the \emph{Hessian} is the matrix
of second-order partial derivatives of $f$.  A point $x\in\mathbb{M}$ is
\emph{critical} when $\nabla f(x)=0$; its value $f(x)$ is a \emph{critical
value}.  Non-critical points and non-critical values are called \emph{regular
points} and \emph{regular values}, respectively.  A level set $f^{-1}(s)$ is
the preimage of a real value $s\in\mathbb{R}$.  If $s$ is regular then 
$f^{-1}(s)$ is a $(d-1)$-manifold.  If $s$ is critical then $f^{-1}(s)$ 
is non-manifold and is called a \emph{critical level set}. 
A critical point with a non-singular Hessian is \emph{non-degenerate}.  
The Morse Lemma states that in the neighborhood of a
non-degenerate critical point $x$, a local coordinate system can be constructed
such that $f(x_0,\ldots, x_n) = f(x) \pm x_0^2 \pm \ldots \pm x_n^2$.  Critical
points are categorized by their \emph{index} which is equal to the number of
minus signs in this equation.  

A function $f$ is \emph{Morse} if it satisfies the two genericity conditions:
1. all critical points are non-degenerate; and 2. $f(x) \ne f(y)$ whenever $x \ne y$ are critical.
The critical points of a Morse function and their indices capture information
regarding the manifold on which the function is defined. For example, the Euler
characteristic of the manifold $\mathbb{M}$ equals the alternating sum of critical
points, $\chi(\mathbb{M}) = \sum_x(-1)^{\mathrm{index}\,x}$.

\begin{figure}[ht]
  \center
  \includegraphics[width=3in]{jpeg/critical.jpg} 
  \caption{Critical points in 1, 2, and 3 dimensions.}
  \label{fig:critical}
\end{figure}

A \emph{triangulation} of a manifold $\mathbb{M}$ is a simplicial complex, $K$, 
whose underlying space is homeomorphic to $\mathbb{M}$~\cite{Alexandrov:Topology}.  
A piecewise linear (PL) function $f$ on $K$ is defined by a set of scalar values
at the vertices that are extended over the simplices of
$K$ via linear interpolation. The function $f$ is assumed to be
non-degenerate and Simulation of Simplicity \cite{Edelsbrunner:SOS} guarantees 
this through symbolic perturbation.
The local structure of $K$ at a vertex is described in terms 
of its \emph{star} and \emph{link}.  The
star of a vertex $u$ consists of all simplices that share $u$, including $u$
itself: $\mathrm{St}\,u  =  \{\sigma \in K \mid u \subseteq \sigma\}$.
The link consists of all faces of simplices in the star that are
disjoint from $u$: $\mathrm{Lk}\,u  =  \{\tau \in K \mid \tau \subseteq \sigma \in
                          \mathrm{St}\,u, u \notin \tau \}$
The lower link is the subset of the link induced by vertices
with function value less than $u$: 
  $\mathrm{Lk}\_\,u  =  \{\tau \in \mathrm{Lk}\,u \mid v \in \tau \Rightarrow
                          f(v) \leq f(u) \} $.

Critical points of piecewise linear functions are classified according to
the reduced Betti numbers \cite{Munkres:Topology} of the lower link, see figure 
\ref{fig:critical}.  
%The k-th reduced Betti number, denoted as $\rBetti_k$,
%is the rank of the k-th reduced homology group of the lower link: $\rBetti_k =
%\mathrm{rank}\,\tilde{\mathsf{H}}_k$. The reduced Betti numbers are the same as
%the usual (un-reduced) Betti numbers, except that $\rBetti_0 = \Betti_0 - 1$
%for non-empty lower links, and $\rBetti_{-1} = 1$ for empty lower
%links~\cite{Munkres:Topology}. When the link is a 2-sphere only $\rBetti_{-1}$ through
%$\rBetti_2$ can be non-zero. Simple critical points have exactly one non-zero
%reduced Betti number, which is equal to unity.
%; see Table~\ref{tab:critClass}.
%\begin{table}[ht]
%\centering
%\begin{tabular}{c || cccc}
% & $\rBetti_{-1}$ & $\rBetti_0$ & $\rBetti_1$ & $\rBetti_2$ \\\hline\hline
%regular  & 0 & 0 & 0 & 0\\\hline
%minimum  & 1 & 0 & 0 & 0\\
%1-saddle & 0 & 1 & 0 & 0\\
%2-saddle & 0 & 0 & 1 & 0\\
%maximum  & 0 & 0 & 0 & 1
%\end{tabular}
%\caption{Classification of vertices into regular and simple critical points
%using the reduced Betti numbers of the lower link.}
%\label{tab:critClass}
%\end{table}
Critical points are assumed to be features in the data and are ranked in
importance by \emph{persistence}. Using persistence, a hierarchical
representation of the data is created in which low persistent features are
identified as noise.  Traditional persistence
\cite{Edelsbrunner:TopoPersistence} is defined as the difference in function
value between pairs of critical points.  

\begin{figure}[ht]
  \center
  \includegraphics[width=2.5in]{jpeg/reeb.jpg} 
  \caption{Left: a height function defined on a 2-D plane, Right: the corresponding merge tree.}
  \label{fig:reeb}
\end{figure}


\paragraph{Mathematical Constructs}
The \emph{Reeb graph} of $f$ is defined by continuously contracting
each contour into a single point.  The Reeb graph consists of a series of nodes
connected by arcs. The nodes correspond to critical points of $f$ on
$\mathbb{M}$, and the arcs correspond to connected components of $\mathbb{M}$, see
figure \ref{fig:reeb}.
In the case of simply connected domains, the Reeb graph has no cycles and is
called a \emph{contour tree}.
Reeb graphs, contour trees, and their variants have been used successfully to guide the
removal of topological features \cite{Cignoni:Simplification, Bremer:Hierarchy, Guskov:NoiseRemoval,
 Carr:Contour, Wood:ExcessTopo}.  In \cite{Mascarenhas:Combustion} contour trees are used to
extract features of interest in large-scale turbulent combustion.
%The work of \cite{Weber:Landscapes} facilitates the understanding of complex scientific information by proposing 
%a terrain metaphor to present the topological information provided by contour trees.

\begin{figure}[ht]
  \center
  \includegraphics[width=5in]{jpeg/morse.jpg} 
  \caption{Left: ascending manifold, Center: descending manifold, Right: MS complex.}
  \label{fig:morse}
\end{figure}

Manifolds can be partitioned according to the behavior of the gradient of $f$ 
rather than the behavior of the level sets of $f$.
An \emph{integral line} is a maximal path on $\mathbb{M}$  whose tangent 
vectors agree with the gradient of $f$.  
A \emph{descending/ascending manifold} of a critical point $u$ is the union of $u$ with 
all integral lines that end/start at $u$.  
The \emph{Morse complex} of $\mathbb{M}$ partitions $\mathbb{M}$
into a set of \emph{descending manifolds}.
The \emph{Morse-Smale complex} (MS complex) is an overlay of ascending and descending 
manifolds which decomposes a scalar field into regions of 
uniform gradient flow behavior, see figure \ref{fig:morse}.  
The MS complex is used in the work of \cite{Laney:TurbulentMixing} 
to analyze Rayleigh-Taylor instabilities by extracting a hierarchical segmentation of the mixing 
envelope surface.  In \cite{Natarajan:Molecules}  molecular surfaces are segmented using the 
MS complex to study semi-rigid structures and the role of cavities and protrusions in protein-protein interactions.
In \cite{Gyulassy:2007:TCD} the MS complex is used to analyze stable channel structures from an atomistic
simulation of a porous solid under impact with a high density projectile. 



%The \emph{Jacobi set}, $\mathbb{J}$ of two Morse functions $f, g$ defined on a common $d$-manifold is 
%the set of critical points of the restrictions of one function to the level 
%sets of the other function. Equivalently, it is the set of points where the 
%gradients of the functions are parallel. Formally,
%$\mathbb{J} = \{ x \in \mathbb{M}\;|\;\nabla f(x) + \lambda\nabla g(x) = 0
%\;\mathbf{or}\;\lambda\nabla f(x) + \nabla g(x) = 0 \}$.
%Jacobi sets are used to describe the path followed by the critical points
%of a time varying function over time, as in \cite{Gyulassy:2007:TCD}, where
%Jacobi sets are used to track channels in a porous solid. 
%In \cite{Edelsbrunner:Func} Jacobi sets are used to study electrostatic potentials of protein
%complexes. 
%For a time-varying function $f:\mathbb{M}\times\mathbb{R}\rightarrow\mathbb{R}$, where
%the last dimension is time, we can introduce an auxiliary function
%$g:\mathbb{M}\times\mathbb{R}\rightarrow\mathbb{R}$ defined as $g(x, t) = t$.  Consider
%the level set $g^{-1}(t) = \mathbb{M}\times t$. The restriction of $f$ to this
%level set is a snapshot of $f$ at time $t$. It follows from the definition that the

